\section{FUNCTIONS, LIMITS, AND CONTINUITY}

\subsection{Scalar and Vector-Valued Functions}

\begin{exercise} \label{e2.1.1}
    Find the parametrizations of each of the following lines:
    \begin{enumerate}
        \item \( 3x_1 + 4x_2 = 6 \)
        
        \item The line with slope \( \frac{1}{3} \) that passes through \( \begin{bmatrix} -1 \\ 2 \end{bmatrix} \)
        
        \item The line through \( A = \begin{bmatrix} 1 \\ 2 \\ 1 \end{bmatrix} \) and \( B = \begin{bmatrix} 2 \\ 1 \\ 0 \end{bmatrix}  \)
        
        \item The line through \( A = \begin{bmatrix} -2 \\ 1 \end{bmatrix} \) perpendicular \( A = \begin{bmatrix} 3 \\ 5 \end{bmatrix}  \)
        
        \item The line through \( \begin{bmatrix} 1 \\ 1 \\ 0 \\ -1 \end{bmatrix} \) parallel to \( \mathbf{g}(t) = \begin{bmatrix} 2+t \\ 1-2t \\ 3t \\ 4-t \end{bmatrix}  \)
    \end{enumerate}
    
    \begin{proof}
        \begin{enumerate}
            \item \[ \mathbf{f}(t) = \begin{bmatrix} t \\ -\frac{3}{4} t + 6 \end{bmatrix} \]
            
            \item \[ \mathbf{f}(t) = t \begin{bmatrix} 1 \\ 3 \end{bmatrix} + \begin{bmatrix} -1 \\ 2 \end{bmatrix} = \begin{bmatrix} t-1 \\ 3t+2 \end{bmatrix} \]
            
            \item \[ \mathbf{f}(t) = t \left( \begin{bmatrix} 1 \\ 2 \\ 1 \end{bmatrix} - \begin{bmatrix} 2 \\ 1 \\ 0 \end{bmatrix} \right) + \begin{bmatrix} 2 \\ 1 \\ 0 \end{bmatrix} \]
            
            \item \[ \mathbf{f}(t) = t \begin{bmatrix} 1 \\ -\frac{3}{5} \end{bmatrix} + \begin{bmatrix} -2 \\ 1 \end{bmatrix} \]
            
            \item \[ \mathbf{g}(t) = \begin{bmatrix} 2+t \\ 1-2t \\ 3t \\ 4-t \end{bmatrix} = t \begin{bmatrix} 1 \\ -2 \\ 3 \\ -1 \end{bmatrix} + \begin{bmatrix} 2 \\ 1 \\ 0 \\4 \end{bmatrix} \]
            Thus
            \[ \mathbf{f}(t) = t \begin{bmatrix} 1 \\ -2 \\ 3 \\ -1 \end{bmatrix} + \begin{bmatrix} 1 \\ 1 \\ 0 \\ -1 \end{bmatrix} \]
        \end{enumerate}
    \end{proof}
\end{exercise}

\begin{exercise} \label{e2.1.2}
    \begin{enumerate}
        \item Give parametric equations for the circle \( x^2 + y^2 = 1 \) in terms of the length \( t \) pictured in the figure.
        
        \item Use your answer to the first part to produce infinitely many positive integer solutions of \( X^2 + Y^2 = Z^2 \) with distinct ratios \( \frac{Y}{X} \)
    \end{enumerate}
    
    \begin{center}
        \begin{tikzpicture}
            \draw[thick] (-1.5,0) -- (1.5,0); % x-axis
            \draw[thick] (0,-1.5) -- (0,1.5); % y-axis
            \draw (0,0) circle(1); % unit circle
            \draw (-1,0) circle(1pt); % line start
            \draw (-1,0) -- (0.6,0.8); % line
            \draw (0.6,0.8) circle(1pt); % line start
            \draw (-1,0) node[anchor=north east] {$\begin{bmatrix} -1 \\ 0  \end{bmatrix}$}; % line start label
            \draw (0.6,0.8) node[anchor=south west] {$\begin{bmatrix} x \\ y  \end{bmatrix}$}; % line end label
            \draw[very thick,red] (0,0) -- (0,0.5); % t length
            \draw[red] (0,0.25) node[anchor=west] {$t$};
            
        \end{tikzpicture}
    \end{center}
    
    \begin{proof}
        \begin{enumerate}
            \item By similar triangles we have that 
            \[ t = \frac{y}{x+1} \]
            which produces the following system:
            \[ \begin{cases} t(x+1) &= y \\ x^2 + y^2 &= 1 \end{cases} \]
            whose solution yields
            \[ \mathbf{f}(t) = \begin{bmatrix} \frac{1-t^2}{1+t^2} \\ \frac{2t}{1+t^2} \end{bmatrix} \]
            
            \item \[ \left\{ \left( 1-t^2, 2t, 1+t^2 \right): t \in \mathbb{Z} \right\} \]
            
            From the figure it is clear that, for \( t > 0 \), \( \frac{Y(t_1)}{X(t_1)} = \frac{Y(t_2)}{X(t_2)} \) iff \( t_1 = t_2 \).
        \end{enumerate}
    \end{proof}
\end{exercise}

\subsection{A Bit of Topology in \( \mathbb{R}^n \)}

\subsection{Limits and Continuity}